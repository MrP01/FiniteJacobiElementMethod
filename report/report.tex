\documentclass{prettytex/ox/mmsc-special-topic}
\input{customisations.tex}
\addbibresource{../literature/sources.bib}
\tikzexternalize[prefix=tikz/]

\newcommand{\topictitle}{
  Finite Element Methods in the Basis of Jacobi Polynomials \\
  \normalsize Subtitle
}
\newcommand{\candidatenumber}{1072462}
\newcommand{\course}{Finite Element Methods}

\title{\topictitle}
\author{Candidate \candidatenumber}
\date{\today}

\makenoidxglossaries
\newacronym{dc}{DC}{Direct Current}
\newacronym{ac}{AC}{Alternating Current}
\newacronym{ode}{ODE}{Ordinary Differential Equation}
\newacronym{pde}{PDE}{Partial Differential Equation}
\newacronym{gui}{GUI}{Graphical User Interface}

\begin{document}
  \pagestyle{plain}
  \mmscSpecialHeader

  \begin{abstract}
    \label{abstract}
    In this project report we will review the central concepts utilised in the group work conducted to make progress in the \gls{pde} problem associated with the electrochemical model of a battery cell and present numerical results \parencite{SchwabCh1998pahf}.
    \vspace*{0.2cm}

    \noindent
    \textbf{Our Goal:}
    Numerically obtain the solution $\{a(x, T), b(x, T)\}$.

    The Finite Difference schemes are implemented in Julia and Python, whereas the Spectral Method is implemented in C++.
  \end{abstract}

  \begin{figure}[H]
    \centering
    % \includegraphics[width=\linewidth]{figures/screenshot.png}
    \caption{The \gls{gui} of the Spectral Solver.}
    \label{fig:gui}
  \end{figure}

  \pagebreak
  \pagestyle{normal}

  \section{Problem Introduction}
  \label{sec:introduction}

  \pagebreak
  \printbibliography
  \printnoidxglossary[type=acronym]

  % \appendix
  % \section{Implementation}
\end{document}
